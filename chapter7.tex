\TUchapter{Conclusions and Future Work}
\TUsection{Conclusions}
\TUsubsection{Summary}
\TUsubsection{Shortcomings}
% Current implementation of time breaks exploit-networkmodel separation, which
% should probably exist, even though it's not stated as a design goal
\TUsection{Related and Future Work}
\TUsubsection{Network Model}
% Hybrid first and third person strategy -- local and adjacent access 
% are topologies; network connection is a quality and therefore assumed
% topology.
\TUsubsection{STRIDE and DREAD}
% qualities as a BAG of CPEs (prefix property)

% SAND/p0f/similar acquisition
\TUsubsection{Exploit Model}
% STRIDE integration (automated mapping onto postconditions?)
\TUsubsection{Exploit Pattern Enhancement}
% regex/wildcard exploits: prefix property for CPEs, 
% general CPE matching: e.g.: cpe:/o:redhat:enterprise_linux:4:ga
%                             cpe:/o:redhat:enterprise_linux:4:.* for any edition
% prefix matching for other facts: e.g. connected matches connected_network, etc.

% AND/OR, etc. possibly by permitting each exploit to have multiple sets of
% preconditions (low-hanging fruit approach).

% variables; this would enable RATE FACTS easing the passage of time (along with
% wildcards, this would make it E-Z!).
\TUsubsection{Hybrid System Equivalence}
% create the equivalence, prove it
\TUsubsection{Hybrid Modeling challenges}
% TIME
%% hard to model like this
%% time-abstraction? (a la time abstract hybrid automata)
\TUsubsection{Model Checking}
\TUsubsection{Analysis and Visualization}
% "proxy anything over ssh" exploit pattern instead of one for each proxied protocol;
% also allow connected_network to suffice for anything, e.g.: 
% connected_(local|adjacent|network(^|_ssh))

% NVD
\TUsubsection{Generation}
% Parallelization/performance
\TUsubsection{Analysis}
\TUchapter{Conclusions and Future Work}
\TUsection{Conclusions}
\TUsubsection{Summary}
This thesis presents a set of expansions to the attack graph formalism
permitting the modeling of hybrid systems in the context of interacting
exploits and components. Several contributions in the modeling domain include
integration with the Common Platform Enumeration, proposal of a standard
lexicon of fact types for easing integration with the Common Vulnerability
Enumeration and the National Vulnerability Database, and dealing with
network objects without loss of generality in the framework.

The hybrid extensions provide the ability to model a small but well-behaved
subset of hybrid systems that nevertheless proves quite expressive. However,
its generation needs effort to measure up to the current state of the art in
performance for attack graph generation.
\TUsubsection{Shortcomings}
\TUsubsubsection{Model acquisition}
Although this thesis takes great strides forward in the articulation of a
language and framework that will be well suited to a wide variety of networks,
a significant roadblock to its use is in network model and exploit pattern
acquisition: although their formats are convenient, they must be automated
before large scale adoption is possible.
\TUsubsubsection{Performance}
It's definitely exponential. More once I've done more science.
\TUsubsubsection{Cognitive scalability}
Perhaps the most severe difficulty with attack graphs is their size and visual
complexity. On their face, they are valuable tools for applying human intuition
to network attack scenarios. However, once the scenario grows beyond a handful
of nodes, the size balloons quickly out of hand. This is even more marked in
time-driven hybrid attack graphs; repeated application of timed exploits that
give little useful data individually further frustrates efforts to promote
cognitive scalability.
\TUsubsubsection{Separation of concerns}
The use of ``exploits'' to implement a system's evolution over time is
unsatisfying. Although it was not stated initially, it is
clear that separating concerns---keeping network model and behavior in the
network model and attacker actions in the exploit patterns---is a useful
design goal especially for the benefit of a modeler.
\TUsection{Future Work}
\TUsubsection{Network Model}
\TUsubsubsection{First and third person modeling}
This thesis used almost exclusively a third person strategy for modeling
attackers and their connection to network assets. In the future, it is worth
exploring the benefits of the first person strategy as well as considering a
blended model in which the Internet (or any network sufficiently large to
wholly cover the domain of the scenario) is implicit and connections to it are
treated as qualities; only local and adjacent access would be implemented as
topologies. This has the potential to ease some of the problems inherent in
chaining together connection topologies across multiple nodes to give the
attacker gradual, intermediated access to a distant host. It may also simplify
exploits.
\TUsubsubsection{Model acquisition}
Significant work needs to be done in the acquisition of models. The integration
of the Common Platform Enumeration eases this; more integration with the
Security Content Automation Protocol (SCAP) may ease this further. % TODO: cite
Nevertheless, some kind of network scanner tool will need to be adapted to
provide automated acquisition of network models in order for attack graph
specification to be automated enough to be a tractable task for larger systems.

Furthermore, such acquisition may be difficult or even impossible in a
hybrid setting. A significant research challenge exists in model acquisition on
the physical side of the cyber-physical divide.
\TUsubsection{Exploit Model}
\TUsubsubsection{Postcondition classification}
There would be value in automated classification of exploit postconditions.
That is, there is a limited number of possible security consequence
classifications of any attack. The National Vulnerability Database contains a
partial attempt to enumerate them. Microsoft makes a similar attempt with
STRIDE. If an exploit can be automatically (or even manually by its source
prior to acquisition for use in the attack graph generator) tagged with its
consequences or potential consequences, this can (1) ease automated creation of
a general exploit pattern and its postconditions, and (2) aid in analysis by
flagging possible consequences of reaching a particular node. 
\TUsubsection{Exploit Pattern Enhancement}
There are a variety of systematic exploit pattern processing enhancements that
would prove useful.
\TUsubsubsection{Wildcard exploit conditions}
% regex/wildcard exploits: prefix property for CPEs exists;
% prefix matching for other facts: e.g. connected matches connected_network, etc.
If exploit preconditions were to permit regular expressions or wildcard
expressions in their statements, this would enable several useful behaviors.
In fact, even permitting the prefix property to apply would help as well. For
example, a precondition of ``connected'' could be satisfied by either
``connected\_network'' or ``connected\_local''. This would permit a hierarchy
of network localities that more accurately reflects the real world.
\TUsubsubsection{Pivot exploits}
On a similar note, a valuable tool for modelers would be generic ``pivot''
exploits. For example, consider the \texttt{ssh\_http\_tunnel} exploit in the
Blunderdome attack graph exploit patterns (see Fig.~\ref{fig:blunder_xp}).
This exploit permits an attacker with certain access to one asset to ``pivot''
from that asset onto all the assets that are connected to it. Specification of
these generic pivot exploits, possibly supported by wildcards in the generator,
would greatly ease the task of a modeler.
\TUsubsubsection{Boolean exploit conditions}
Permitting boolean operators in exploit preconditions would be valuable, as
otherwise (currently) multiple exploits with the same sets of postconditions but
slightly varied preconditions must be used to represent exploits that may
affect, for example, Linux version 2.6.33.19, 2.6.34.10, and 2.6.38.8.
\TUsubsubsection{Variable exploit conditions}
Currently, the right hand side of exploit conditions must be literals. If they
could be expressions involving variables loaded from the fact base, this would 
satisfy the required conditions for linear hybrid automata equivalence (see
Section~\ref{sec:hag_ha}.
\TUsubsection{Hybrid Modeling challenges}
Time is difficult to model and interpret when implemented in this way. One
alternative may be to look again to the hybrid automata for inspiration and
adopt a type of transition similar to its \emph{time-abstract} semantics,
which indicate \emph{that} time has passed, but not precisely \emph{how much}
has passed. This kind of implementation would eliminate much of the size and,
potentially, generation scaling issues with this proposed framework.
\TUsubsection{Generation}
\TUsubsubsection{Performance}
The generation performance of the hybrid attack graph generation needs to be
brought into step with the current state of the art. This was not the goal of
this particular work, but it is in important step in its future application.
\TUsubsubsection{Parallelization}
One route to improving performance that would be of use to the wider world is
the parallelization of the generation procedure. As scaling remains an issue
in attack graph generation in general, parallelizing the process could permit
greater scaling and possibly allow attack graph analysis to grow to the scale
of large enterprise networks.
\TUsubsection{Attack Dependency Graphs}
Finally, one route for future work that deserves specific emphasis is the recently
articulated attack dependency graph (also called the exploit dependency graph).
A variation of the attacks graph adopted by some researchers, attack dependency
graphs build graphs of the dependencies of attacks upon each 
other~\cite{jajodia2005topological, noel2004managing}. These smaller models may
be more efficiently generated, encode equivalent information, and eliminate 
visual complexity.

Attack dependency graphs address several of the shortcomings of the stateful
hybrid attack graphs described in this chapter. They are more efficient to
generate, easing performance concerns. Furthermore, the hybrid versions proposed
in a University of Tulsa abstract~\cite{louthan2011toward} exhibit the
desired time abstract behavior and separation of concerns mentioned above. These
may very well be the best direction for future work in hybrid attack graph
study to proceed.

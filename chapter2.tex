\chapter{Background}
\section{Cyber Physical Systems}
\subsection{Hybrid Systems}
A system with both continuous (frequently physical) components and discrete (frequently digital)
components is said to be a \emph{hybrid system}, named for its characteristic blending of the
two domains. Examples of hybrid computer systems abound in industrial controls, for example,
although hybrid systems may also be fully physical (e.g., a bouncing ball experiences continuous
behavior when rising and falling and discrete behavior when colliding with a surface).

The term hybrid system is an older one that was coined as researchers began to study the newly
pervasive reactive systems that arose as programmed control of the physical world became 
widespread ~\cite{alur1993hybrid}. For several reasons it does not suffice to describe precisely
the types of systems with which this work is concerned: a subset of hybrid
systems that incorporate a significant computer and networking component.

Nevertheless, the modeling of hybrid systems is well studied and provides a sufficient body
of relevent knowledge from which to draw to warrant its inclusion. This chapter includes
background on a particularly relevent modeling framework for hybrid systems called the
hybrid automaton, which is used in this thesis as the standard benchmark against which to
compare hybrid modeling techniques.
\subsection{Cyber Physical Systems}
\subsubsection{Definition}
% Definition
A newer, better term for the systems investigated in this thesis is \emph{cyber physical systems}.
Put simply, a cyber physical system is a networked hybrid system: a networked computer system that is 
tightly coupled to the physical world. 
\subsubsection{Challenges}
According to the 2008 Report of the Cyber-Physical Systems Summit, ``The principal barrier to 
developing CPS is the lack of a theory that comprehends cyber and physical resources in a 
single unified framework.''~\cite{summitreport2008}

The summit further identified as part of the scientific and technological foundations of
cyber physical systems both new modeling frameworks that ``explicitly address new observables'' and 
studies of privacy, trust, and security including ``theories of cyber-physical inter-dependence''~\cite{summitreport2008},
a major theme of this work.

Crenshaw and Beyer enumerated four principal challenges in cyber physical systems testing that are
equally apt for security:
their concentration in safety critical domains, their frequent integration of third-party or
otherwise unrelated systems, their dependence upon unreliable data collection, and their
pervasity~\cite{crenshaw2010upbot}.
\subsection{Hybrid Automata}
\subsubsection{Definition}
A valuable formalism for modeling hybrid systems in isolation and with limited composition
is the hybrid automaton of Alur, et al.~\cite{alur1993hybrid}. This section introduces the
version of the formalism described in 1996 by Henzinger~\cite{henzinger1996theory}, to which a
reader interested in more than a superficial understanding is referred. 

Formally,
a hybrid automaton $H$ is made up of a set of real-valued state variables, their first derivatives,
a set of operational modes and switches between the modes, and predicates attached to those modes and
switches describing the operation of the system in those modes and the discrete transitions between
them. 
\subsubsection{Shortcomings}
\subsubsection{Alternatives}
\section{Attack Graphs}
\section{Case Studies}
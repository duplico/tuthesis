\TUchapter{Introduction}
\TUsection{Introduction}
As computer systems become pervasive, not only are 
their interactions with people becoming more frequent; they are also
increasingly interacting with the physical world and with each other. These
systems sometimes bridge the divide between the computational and the physical,
blending discrete and continuous elements into a single system.

Such systems are termed \emph{hybrid systems}. When linked together with a 
significant network component, these systems are sometimes called
\emph{cyber-physical systems}. They are pervasive in 
safety-critical applications such as medical, critical infrastructure, and
automotive equipment. This thesis is concerned with modeling
these systems, their security, and the manner in which their components
interact with each other and the physical world.
\TUsection{Modeling Frameworks}
An excellent argument for the need for new research in modeling cyber-physical 
systems is due to Lee:
\begin{quote}
Cyber-Physical Systems (CPS) are integrations of computation with physical
processes. Embedded computers and networks monitor and control the physical
processes, usually with feedback loops where physical processes affect computations 
and vice versa. In the physical world, the passage of time is inexorable
and concurrency is intrinsic. Neither of these properties is present in today's
computing and networking abstractions~\cite{lee2006cyber}.
\end{quote}

This still holds true today. Existing frameworks---abstractions---for modeling 
and analysis of purely discrete computer networks are inappropriate for use in
these systems because of their inability to capture the continuous domain; they
also lack a robust, let alone ``inexorable,'' notion of time. Likewise, 
modeling methods from the world of isolated control systems cannot model the
complex distributed networks that are the hallmark of cyber-physical systems.

\TUsection{Scope}
This thesis presents an extension of the attack graph into the 
continuous domain. 
The goal is to incorporate aspects of hybrid systems
modeling (e.g. hybrid automata), which best describe systems in 
relative isolation into attack graphs, which excel at capturing complex 
interrelationships and interdependencies among systems and attacks.

The remainder of this thesis is structured as follows. Chapter 2 provides 
background in hybrid systems and their modeling, introduces past work in attack 
graphs, and presents a set of case studies in both the hybrid and discrete 
domains used throughout this work. Chapter 3 contributes a specification for 
the generation process. Chapter 4 introduces in detail an expanded attack graph 
model to be used as the basis for the hybrid extensions. Chapter 5
introduces those extensions themselves. Chapter 6 delivers some results from 
this modeling methodology, and Chapter 7 draws conclusions and suggests 
further work.
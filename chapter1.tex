\chapter{Introduction}
\section{Introduction}
As computer systems become pervasive across a variety of domains, not only are their interactions
with people becoming more frequent; computer systems are also increasingly interacting with the 
physical world and with each other.

Systems that include both continuous and discrete components are termed \emph{hybrid systems}.
When linked together with a significant network component, these systems are sometimes called
\emph{cyber physical systems}, which have been targeted as a key area of research. Such systems
are becoming pervasive in safety-critical domains such as medical, critical infrastructure, 
automotive, and others. This thesis is concerned with modeling the security of these systems and
their interactions with each other and the physical world.
\section{Modeling Frameworks}
An excellent argument that touches on the need for new research directions in modeling cyber physical systems
is due to Lee in a 2006 position paper in the National Science Foundation Workshop on Cyber-Physical Systems,
a prelude to the NSF's research initiative on cyber physical systems:
\begin{quote}
Cyber-Physical Systems (CPS) are integrations of computation with physical
processes. Embedded computers and networks monitor and control the physical
processes, usually with feedback loops where physical processes affect compu-
tations and vice versa. In the physical world, the passage of time is inexorable
and concurrency is intrinsic. Neither of these properties is present in today's
computing and networking abstractions. ~\cite{lee2006cyber}
\end{quote}

Existing frameworks for modeling and analysis of computer networks are inappropriate for use in
these systems because of their inability to capture the continuous domain; they
also lack a robust, let alone ``inexorable'' notion of time. Likewise, existing
methods for studying hybrid systems fall short when it comes to modeling the sometimes complex
networks that are hallmarks of cyber physical systems.
\section{Scope}
This thesis presents an extension of the attack graph modeling framework, typically used for studying
network security, into the continuous domain to enable it to be used for studying cyber
physical systems. The goal is to combine the best of both worlds: hybrid systems
modeling frameworks, particularly hybrid automata, which best describe systems in 
relative isolation; and computer network security modeling frameworks, particularly attack graphs,
which excel at capturing the complex interrelationships and interdependencies between assets and
attacks.

The remainder of this thesis is structured as follows. Chapter 2 provides background in hybrid systems
and their modeling methods, introduces past work in attack graphs, and presents a set of case studies
in both the hybrid and discrete domains to be used throughout this work. Chapter 3 introduces in detail
the basic attack graph framework to be used as the basis for the hybrid extensions. Chapter 4
introduces the extensions themselves. Chapter 5 delivers some results from this modeling methodology,
and Chapter 6 draws conclusions and suggests further work.
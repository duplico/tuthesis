\chapter{Introduction}
\section{Introduction}
As computer systems become pervasive across a variety of domains, not only are their interactions
with people becoming more frequent; computer systems are also increasingly interacting with the 
physical world and with each other.
\section{Hybrid Systems}
Systems that include both continuous and discrete components are termed \emph{hybrid systems}.
When linked together with a significant network component, these systems are sometimes called
\emph{cyber physical systems}, which have been targeted as a key area of research. Such systems
are becoming pervasive in safety-critical domains such as medical, critical infrastructure, 
automotive, and others, and their security is an important area of research and the topic of this
thesis.
\section{Modeling Frameworks}
Existing frameworks for modeling and analysis of computer networks are inappropriate for use in
these systems because of their inability to capture the continuous domain. Likewise, existing
methods for studying hybrid systems fall short when it comes to modeling the sometimes complex
networks that are hallmarks of cyber physical systems.

\section{Scope}
This thesis presents an extension of the attack graph modeling framework, typically used for studying
network security, into the continuous domain to enable it to be used for studying cyber
physical systems.

The remainder of this thesis is structured as follows. Chapter 2 provides background in hybrid systems
and their modeling methods, introduces past work in attack graphs, and presents a set of case studies
in both the hybrid and discrete domains to be used throughout this work. Chapter 3 introduces in detail
the basic attack graph framework to be used as the basis for the hybrid extensions. Chapter 4
introduces the extensions themselves. Chapter 5 delivers some results from this modeling methodology,
and Chapter 6 draws conclusions and suggests further work.
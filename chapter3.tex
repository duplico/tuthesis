
\TUsubsection{Generation Algorithm}
\TUsubsection{Output}
In order to aid understanding the generation process, this section expands on the
illustrative example introduced in the preceding sections and demonstrates a
sample attack graph generation process using the example. The reader is warned to
avoid searching for meaning or design in the selection of these assets, qualities,
and topologies; they are intended to be illustrative only, and any relation to 
real world networks, problems, attacks, or situations is purely coincidental.
This section will use the following network model specification.
\begin{lstlisting}
network model = 
    assets :
        asset_1;
        asset_2;
        asset_3;

    facts :
        quality:asset_1,quality_1,value_1;
        quality:asset_2,quality_1,value_2;        
        quality:asset_3,quality_1,value_2;
        topology:asset_1,asset_2,topology_1;
        topology:asset_2,asset_3,topology_2;        
        topology:asset_3,asset_2,topology_2;
.
\end{lstlisting}

The following exploit pattern specifications are used in this section.
\begin{lstlisting}
exploit exploit_1(asset_param_1,asset_param_2)=
    preconditions:
        quality:asset_param_1,quality_1,value_1;
        topology:asset_param_1,asset_param_2,topology_1;
    postconditions:
        delete topology:asset_param_1,asset_param_2,topology_1;
        insert quality:asset_param_1,quality_1,value_2;
.

exploit exploit_2(asset_param_1,asset_param_2)=
    preconditions:
        quality:asset_param_1,quality_1,value_2;
        quality:asset_param_2,quality_1,value_2;
    postconditions:
        insert topology:asset_param_1,asset_param_2,topology_2;
        insert topology:asset_param_2,asset_param_1,topology_2;
.
\end{lstlisting}

Fig.~\ref{fig:ill_topology_1} represents the initial network state 
(denoted State 1) specified
in this example. Note that Fig.~\ref{fig:ill_topology_1} is \emph{not} an attack
graph, merely a convenient graph based representation of the example network
in use here. Since there is only one quality in use in the entire example, its
name is not included in this graph representation; instead, each asset is labeled
with its identifier and the value of \texttt{quality\_1} with the prefixing 
\texttt{value\_} removed for brevity. Likewise, the edges that represent topologies
are labeled with the topology name they represent with the prefixing \texttt{topology\_}
removed.
\begin{figure}
\centering
\begin{dot2tex}[options=-t raw --autosize]
digraph G {
    rankdir=LR;
    asset_1 [shape=circle, texlbl="\begin{tabular}{c}\texttt{\bf asset\_1} \\ $1$ \end{tabular}"];
    asset_2 [shape=circle, texlbl="\begin{tabular}{c}\texttt{\bf asset\_2} \\ $2$ \end{tabular}"];
    asset_3 [shape=circle, texlbl="\begin{tabular}{c}\texttt{\bf asset\_3} \\ $2$ \end{tabular}"];
	asset_1 -> asset_2 [label=" ", texlbl="1"];
    asset_2 -> asset_3 [label=" ", texlbl="2"];
    asset_3 -> asset_2 [label=" ", texlbl="2"];
}
\end{dot2tex}
\caption{State 1 (initial network state) of the illustrative discrete example}
\label{fig:ill_topology_1}
\end{figure}

Execution of the generation process begins by specifying a maximum ``depth''
of generation, which will be 3 for the purposes of this exercise (although,
as this attack graph converges with maximum shortest path of 3 from the
initial state, this limitation is unnecessary), and by creating an initial list
of states for analysis, which contains only State 1.

Generation of the next set of states begins by creating a list of all valid
attacks on State 1; that is, selecting all valid bindings of assets given State 1's
fact base to exploit pattern parameters given their preconditions. Three such bindings
are possible from State 1: \texttt{exploit\_1(asset\_1, asset\_2)}; \texttt{exploit2(asset\_2, asset\_3)}; and
\texttt{exploit2(asset\_2, asset\_3)}. The latter two insert a topology into the fact base
that already exists, therefore generating State 1, so State 1 has itself as a successor state
twice. As State 1 already exists, these edges are added, and no further processing due to
those transitions is done. The first attack, \texttt{exploit\_1(asset\_1, asset\_2)}, results in
a new state, designated State 2. It is generated by performing the operations on State 1's
fact base and creating a new state based on the results. State 2 is the only state added to
the list of successor states.

\begin{figure}
\centering
\begin{dot2tex}[options=-t raw --autosize]
digraph G {
    rankdir=LR;
    asset_1 [shape=circle, texlbl="\begin{tabular}{c}\texttt{\bf asset\_1} \\ $2$ \end{tabular}"];
    asset_2 [shape=circle, texlbl="\begin{tabular}{c}\texttt{\bf asset\_2} \\ $2$ \end{tabular}"];
    asset_3 [shape=circle, texlbl="\begin{tabular}{c}\texttt{\bf asset\_3} \\ $2$ \end{tabular}"];
    asset_2 -> asset_3 [label=" ", texlbl="2"];
    asset_3 -> asset_2 [label=" ", texlbl="2"];
}
\end{dot2tex}
\caption{State 2 of the illustrative discrete example}
\label{fig:ill_topology_2}
\end{figure}

For the next execution of the generation function, the successor state list becomes
the new analysis state list, consisting now only of State 2, and the 
remaining allowed ``depth'' is decremented to 2. 6 attacks are possible from State 2:
\texttt{exploit\_2(asset\_1, asset\_2)}, \texttt{exploit\_2(asset\_2, asset\_1)}, 
\texttt{exploit\_2(asset\_1, asset\_3)}, \texttt{exploit\_2(asset\_3, asset\_1)},
\texttt{exploit\_2(asset\_2, asset\_3)}, and \texttt{exploit\_2(asset\_3, asset\_2)}. The process
iterates through these attacks, generating new states and, if they are new, adding them
to the list of successor states. \texttt{exploit\_2(asset\_1, asset\_2)} results in a new
state, State 3 (see Fig.~\ref{fig:ill_topology_3}), which is added to the list of
successor states. \texttt{exploit\_2(asset\_2, asset\_1)} also results in State 3, which
already exists and is therefore not added to the list of successor states. 
\texttt{exploit\_2(asset\_1, asset\_3)} generates a new state, State 4 
(see Fig.~\ref{fig:ill_topology_4}), which is added to the list of successor states.
\texttt{exploit\_2(asset\_3, asset\_1)} also generates State 4.
\texttt{exploit\_2(asset\_2, asset\_3)} and \texttt{exploit\_2(asset\_3, asset\_2)}
both result in State 2, which exists and is not added to the successor states.

\begin{figure}
\centering
\begin{dot2tex}[options=-t raw --autosize]
digraph G {
    rankdir=LR;
    asset_1 [shape=circle, texlbl="\begin{tabular}{c}\texttt{\bf asset\_1} \\ $2$ \end{tabular}"];
    asset_2 [shape=circle, texlbl="\begin{tabular}{c}\texttt{\bf asset\_2} \\ $2$ \end{tabular}"];
    asset_3 [shape=circle, texlbl="\begin{tabular}{c}\texttt{\bf asset\_3} \\ $2$ \end{tabular}"];
    asset_2 -> asset_3 [label=" ", texlbl="2"];
    asset_1 -> asset_2 [label=" ", texlbl="2"];
    
    asset_3 -> asset_2 [label=" ", texlbl="2"];
    asset_2 -> asset_1 [label=" ", texlbl="2"];
}
\end{dot2tex}
\caption{State 3 of the illustrative discrete example}
\label{fig:ill_topology_3}
\end{figure}

\begin{figure}
\centering
\begin{dot2tex}[options=-t raw --autosize]
digraph G {
    rankdir=LR;
    asset_1 [shape=circle, texlbl="\begin{tabular}{c}\texttt{\bf asset\_1} \\ $2$ \end{tabular}"];
    asset_2 [shape=circle, texlbl="\begin{tabular}{c}\texttt{\bf asset\_2} \\ $2$ \end{tabular}"];
    asset_3 [shape=circle, texlbl="\begin{tabular}{c}\texttt{\bf asset\_3} \\ $2$ \end{tabular}"];
    asset_2 -> asset_3 [label=" ", texlbl="2"];
    asset_1 -> asset_3 [label=" ", texlbl="2"];
    
    asset_3 -> asset_2 [label=" ", texlbl="2"];
    asset_3 -> asset_1 [label=" ", texlbl="2"];
}
\end{dot2tex}
\caption{State 4 of the illustrative discrete example}
\label{fig:ill_topology_4}
\end{figure}

The next iteration finds the remaining allowed depth at 1 and the list of
analysis states to contain State 3 and State 4. From State 3, the
possible attacks are the same as the possible attacks in State 2:
\texttt{exploit\_2(asset\_1, asset\_2)}, \texttt{exploit\_2(asset\_2, asset\_1)},
\texttt{exploit\_2(asset\_1, asset\_3)}, \texttt{exploit\_2(asset\_3, asset\_1)},
\texttt{exploit\_2(asset\_2, asset\_3)}, and \texttt{exploit\_2(asset\_3, asset\_2)}.
Only the middle two, \texttt{exploit\_2(asset\_1, asset\_3)}, \texttt{exploit\_2(asset\_3, asset\_1)},
generate a state that is not State 3 itself: State 5 (see Fig.~\ref{fig:ill_topology_5}),
which is added to the list of successor states. The rest simply create new reflexive
edges on State 3. The list of possible attacks on State 4 is the same, with only
\texttt{exploit\_2(asset\_1, asset\_2)} and \texttt{exploit\_2(asset\_2, asset\_1)} generating
a state other than State 4 itself. Both of those attacks generate State 5 again, which
already exists and therefore is not added to the list of successor states.

\begin{figure}
\centering
\begin{dot2tex}[options=-t raw --autosize]
digraph G {
    rankdir=LR;
    asset_1 [shape=circle, texlbl="\begin{tabular}{c}\texttt{\bf asset\_1} \\ $2$ \end{tabular}"];
    asset_2 [shape=circle, texlbl="\begin{tabular}{c}\texttt{\bf asset\_2} \\ $2$ \end{tabular}"];
    asset_3 [shape=circle, texlbl="\begin{tabular}{c}\texttt{\bf asset\_3} \\ $2$ \end{tabular}"];
    asset_1 -> asset_2 [label=" ", texlbl="2"];
    asset_2 -> asset_3 [label=" ", texlbl="2"];
    asset_3 -> asset_1 [label=" ", texlbl="2"];
    
    asset_2 -> asset_1 [label=" ", texlbl="2"];
    asset_3 -> asset_2 [label=" ", texlbl="2"];
    asset_1 -> asset_3 [label=" ", texlbl="2"];
}
\end{dot2tex}
\caption{State 5 of the illustrative discrete example}
\label{fig:ill_topology_5}
\end{figure}

The next iteration finds the list of analysis states to contain only State 5, and
the remaining allowed depth is 0. Because the depth limit is reached, generation
ceases. However, even if the depth limit had not been reached, execution would have
soon terminated. All 6 possible attacks from State 5 would have generated State 5 itself.
Therefore the list of successor states would be empty, and the next iteration's list of
analysis states would be empty, which would also cause generation to cease. A representation
of the resulting attack graph (with execution halting after 3 iterations) is provided
in Fig.~\ref{fig:ill_attack_graph}.

\begin{figure}
\centering
\begin{dot2tex}[options=-t raw --autosize]
digraph G {
    rankdir=LR;
    state_1 [shape=circle, texlbl="State 1"];
    state_2 [shape=circle, texlbl="State 2"];
    state_3 [shape=circle, texlbl="State 3"];
    state_4 [shape=circle, texlbl="State 4"];
    state_5 [shape=circle, texlbl="State 5"];
    
    state_1 -> state_1;
    state_1 -> state_1;
    state_1 -> state_2;
    
    state_2 -> state_2;
    state_2 -> state_2;
    state_2 -> state_3;
    state_2 -> state_3;
    state_2 -> state_4;
    state_2 -> state_4;
    
    state_3 -> state_3;
    state_3 -> state_3;
    state_3 -> state_3;
    state_3 -> state_3;
    state_3 -> state_5;
    state_3 -> state_5;
    
    state_4 -> state_4;
    state_4 -> state_4;
    state_4 -> state_4;
    state_4 -> state_4;
    state_4 -> state_5;
    state_4 -> state_5;
}
\end{dot2tex}
\caption{The illustrative example's attack graph to depth 3}
\label{fig:ill_attack_graph}
\end{figure}

This concludes the description of the basic attack graph model employed at the
University of Tulsa. The chapters following this one are devoted to
this model's expansion and analysis.
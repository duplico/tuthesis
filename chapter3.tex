\TUchapter{Attack Graphs}
\TUsection{Introduction}
This chapter presents a version of the attack graph modeling framework specific
to traditional information systems. The framework has a very permissive model that
includes notions of assets, qualities, topologies. Assets represent potentially
attackable system components; qualities assign an arbitrary string value to a named
property of an asset, and topologies bind pairs of assets together with a named
connection between them. Exploit patterns with preconditions and postconditions attached
to free variables that can be bound at generation time to assets serve to describe potential
state transitions. The next section contains a more formal definition of the basic
framework.

Because of the extremely permissive model for naming properties and topologies in this
scheme, the modeler must fix a lexicon of terms and their meanings in order to proceed
systematically. Section 3 of this chapter presents a preliminary version of such a lexicon,
fixing enough terminology to proceed with the case studies.

Research into attack graphs can be split into four broad categories:
\begin{description}
\item[Modeling] Attack graph modeling concerns the development of the underlying representation
	and use of that representation to model systems. Terminology belongs here, as do efforts to
	automatically generate network models from real networks and exploit patterns from vulnerability
	databases.
\item[Generation] Attack graph generation is the process of building a graph out of a model by
	closing the state space over its exploits. This is where most performance work is concentrated.
	A good portion of the work that enables a representation of time to be included in attack graphs
	belongs here as well. Constraints (such as monotonicity) on the way that state transitions are 
	allowed to progres also fall under the generation category.
\item[Analysis] Analysis of attack graphs focuses on drawing conclusions about a system based upon
	the attack graph generated from its model. Work here includes integration with intrusion detection
	systems, automatic delivery of mitigation recommendations, and the identification of security
	consequences with particular states.
\item[Visualization] Visualization of attack graphs seeks to reduce or eliminate their
	known cognitive scalability issues; the goal is to deliver the results of the other three
	steps in a meaningful fashion.
\end{description}
This thesis is mainly concerned with the modeling stage, its goal being to introduce an
architecture for modeling hybrid systems with attack graphs. However, some amount of work is
included in the generation stage, particularly some thoughts on dealing with the progression
of time; consideration is also given to analysis, particularly on identifying failure states and
on aggregating sufficiently similar states (which also touches visualization).

Sections 4 and 5 introduce two analysis tools for attack graphs: state predicates and state
aggregation. State predicates are logical predicates concerning a network state that could
be used to specify particular failure cases, nominal system behavior, operating modes,
or similarity conditions that permit states to be visually grouped due to some shared characteristics.
This concept is necessary to allow the following chapter's introduction of time to be tractable.
\TUsection{Definition}
\TUsubsection{Intuitive}
For the purposes of this work, an attack graph is comprised of the following components.
\begin{description}
	\item[Assets] Assets are the subjects in the attack graph formalism, mainly representing
		attackable system components. For example, in a model of the RFID denial of sleep
		attack, the assets are the RFID tags, the reader, the possible rogue reader,
		and any hosts connected over the network with the RFID reader. Assets are specified
		with unique names, and they are decorated with \emph{qualities} and \emph{topologies}.
		Assets can also be used to model users and adversaries if it is necessary to give them
		explicit properties. A model's collection of assets is fixed at definition time and
		is not changed by state transitions.
	\item[Qualities] Qualities represent properties of an asset, such as a software package or
		version that is installed, whether it is in sleep mode or not, and so on. Qualities are
		can be considered a key/value pair, where both the key and the value are string tokens. For
		example, the \texttt{host1} asset may have the quality \texttt{power} and the value \texttt{on}.
		Together with topologies, qualities make up the collection of facts.
	\item[Topologies] Topologies represent relationships between two assets. These can be physical such
		as denoting that a printer is plugged into a computer, logical such as denoting that one
		host is accessible from another on an adjacent network, or more abstract such as a particular
		trust relationship or level of access that a subject has on another. Topologies are directed and
		named with string tokens. For example, \texttt{host1} might be accessible over the network via the
		web by \texttt{host2}, so a directed topology from \texttt{host2} to \texttt{host1} called
		\texttt{network\_remote\_web} might be used. Together with qualities, topologies make up the
		network state's collection of facts.
	\item[State] A network or system state is comprised of all of the facts about the system's asset
		collection. A network model's state is fully described by the asset collection and the fact base;
		given the constant asset collection, a state is uniquely described by its fact base. The fact
		base is all the qualities and topologies that are valid for that state.
	\item[Exploit patterns] Exploit patterns are generalized templates for how the actions of the attacker
		can alter the system state by inserting and removing qualities and topologies (but not assets).
		They are written as functions that take a number of parameters corresponding to assets, mapping
		a set of preconditions (facts about the free asset parameters) to a set of postconditions:
		insert and delete actions on qualities and topologies to update the fact base and therefore
		generate a new network state.
\end{description}
\TUsubsection{Formal}
\TUsubsubsection{Primitive Domains}
In order to provide a more formal description of the attack graph modeling framework,
it is necessary to fix first the domains of note. The primitive domains are the most basic 
domains that describe the atomic units of
the formalism: assets, properties (qualities), values (qualities), relationships (topologies),
vulnerabilities (exploit pattern identifiers), and parameters (free asset variables in the attack
patterns), and operations (used in postconditions representing insert or delete actions):
\begin{align*}
    \mathcal{A} :& \text{assets} \\
    \mathcal{P} :& \text{properties (quality names)} \\
    \mathcal{V} :& \text{values (property values)} \\
    \mathcal{R} :& \text{relationships (topology names)} \\
    \mathcal{W} :& \text{vulnerabilities (exploit pattern names)} \\
    \mathcal{I} :& \text{parameters (free asset names)} \\
    \text{Op} =& \left\{\text{ins}, \text{del} \right\}
\end{align*}

\TUsubsubsection{Compound Domains}
Compound domains comprise the fundamental concepts that are composed of combinations
of members of the primitive domains. These form the level of abstraction that it is
most convenient to discuss in the articulation of the execution model and in the
preceding intuitive definition, for instance.
\begin{description}
    \item[Qualities] Qualities bind an asset to a property to a value; therefore their
        domain is the cartesian product of those domains. The $n$ subscripts denote that
        these are bound qualities of a network state, rather than free qualities in
        exploit preconditions and postconditions:
        \begin{align*}
            \mathcal{Q}_n &: \mathcal{A} \times \mathcal{P} \times \mathcal{V}
        \end{align*}
    \item[Topologies] Topologies bind an asset to another asset through a relationship; therefore their
        domain is the cartesian product of those domains:
        \begin{align*}
            \mathcal{T}_n&: \mathcal{A} \times \mathcal{A} \times \mathcal{R}
        \end{align*}
    \item[Network states] A network state, then, is denoted by a collection of assets, and a fact base
        of qualities and topologies; the domain of a network state is the cartesian product
        of the power sets of these domains:
        \begin{align*}
            \mathcal{N}&: \mathbb{P}(\mathcal{A}) \times \mathbb{P}(\mathcal{Q}_n) \times \mathbb{P}(\mathcal{T}_n)
        \end{align*}
    \item[Exploit patterns] Exploit patterns, taken from the domain $\mathcal{E}$ 
		depend upon free versions of qualities and topologies,
		which are parameterized by members of $\mathcal{I}$ rather than $\mathcal{A}$, which are used in
		preconditions and, when combined with an operator, postconditions:
        \begin{align*}
			\mathcal{Q}_e&: \mathcal{I} \times \mathcal{P} \times \mathcal{V} \\
			\mathcal{T}_e&: \mathcal{I} \times \mathcal{I} \times \mathcal{R} \\
			\text{Preconditions } \mathcal{P}rc_e &: \mathbb{P}(\mathcal{Q}_e) \times \mathbb{P}(\mathcal{T}_e) \\
			\text{Postconditions } \mathcal{P}oc_e&: \mathbb{P}((Op,\mathcal{Q}_e)) \times \mathbb{P}((Op,\mathcal{T}_e)) \\
			\mathcal{E}&: \mathcal{W} \times \vec{\mathcal{I}} \times  \mathcal{P}rc \times \mathcal{P}oc
        \end{align*}
    \item[Attacks] An exploit pattern whose parameters have been bound to assets is referred to as an
		attack. It takes a similar appearance:
		\begin{align*}
			\text{Preconditions } \mathcal{P}rc_n &: \mathbb{P}(\mathcal{Q}_n) \times \mathbb{P}(\mathcal{T}_n) \\
			\text{Postconditions } \mathcal{P}oc_n&: \mathbb{P}((Op,\mathcal{Q}_n)) \times \mathbb{P}((Op,\mathcal{T}_n)) \\
			\mathcal{X}&: \mathcal{W} \times \vec{\mathcal{I}} \times  \mathcal{P}rc \times \mathcal{P}oc
        \end{align*}
\end{description}
% TODO - fill in the rest of this.
\TUsection{Execution Model}
\TUsubsection{Introduction}
The reference implementation of this thesis's attack graph execution model does not use this
formal notation to represent system elements, instead favoring a more user friendly specification
language. This section describes that specification language and the generation process. % Todo: name it? :)
\TUsubsection{Network Model Specification}
A network model specification consists of a list of assets and an initial fact base 
(list of qualities and topologies). It begins with the phrase \texttt{network model}, followed
by the~\texttt{=} symbol. 

The first component of the specification itself is an asset list.
The asset list is a semicolon separated list of the names of network
assets preceded by the word \texttt{assets} and a colon. For example:
\begin{lstlisting}
network model = 
assets :
asset_1;
asset_2;
asset_3;
\end{lstlisting}

Following the asset list, the initial
fact base is specified as a semicolon separated list of facts freceded by the word
\texttt{facts} and a colon. Facts may occur in any order. Qualities are specified by the
word \texttt{quality}, followed by a colon, followed by the asset in question, a comma,
then the name of the quality, then the~\texttt{=} symbol, then the value of the quality.
Topologies are specified by the word~\texttt{topology}, followed by a colon, followed by the names
of the assets in question separated by a directionality symbol:~\texttt{->} to represent a one-way
topology from the left asset to the right asset, or~\texttt{<->} to represent a two-way topology.
This is followed by a comma, then the name of the topology. The fact listing, and thus the
network model specification, is concluded with a period. For example:
\begin{lstlisting}
facts :
	quality:asset_1,quality_1=value_1;
	quality:asset_2,quality_1=value_2;
	topology:asset_1->asset_2,topology_1;
	topology:asset_2<->asset_3,topology_2;
.
\end{lstlisting}

The completed version of this network model specification would be as follows.
\begin{lstlisting}
network model = 
    assets :
    asset_1;
    asset_2;
    asset_3;

    facts :
        quality:asset_1,quality_1=value_1;
        quality:asset_2,quality_1=value_2;
        topology:asset_1->asset_2,topology_1;
        topology:asset_2<->asset_3,topology_2;
.
\end{lstlisting}
\TUsubsection{Exploit Specification}
Exploits are specified using a related scheme. An exploit pattern resembles a function
and contains four parts: a header or signature, preconditions, postconditions, and a
terminating period symbol.

An exploit specification begins with the word \texttt{exploit} followed by a
unique identifier to name the exploit pattern, an opening parenthesis, a comma separated
list of parameter names (which are to be bound to assets), a closing parenthesis, and 
the~\texttt{=} symbol. For example:
\begin{lstlisting}
exploit exploit_1(asset_param_1,asset_param_2)=
\end{lstlisting}

The preconditions list follows. It begins with the word \texttt{preconditions}, followed
by a colon, followed by a semicolon separated sequence of facts, which adhere to the
same grammar as the facts for network model specification. For example:
\begin{lstlisting}
preconditions:
    quality:asset_param_1,quality_1=value_1;
    topology:asset_param_1->asset_param_2,topology_1;
\end{lstlisting}

The postconditions list follows this. It begins with the word \texttt{postconditions},
followed by a colon, followed by a semicolon separated sequence of operations. Operations
consist of the word \texttt{insert} or \texttt{delete}, followed by a space, followed by
a fact. For example:
\begin{lstlisting}
postconditions:
    delete topology:asset_param_1->asset_param_2,topology_1;
    insert quality:asset_param_1,quality_1=value_2;
\end{lstlisting}

The exploit pattern is terminated with a period character. The full example from this
subsection would appear as follows.
\begin{lstlisting}
exploit exploit_1(asset_param_1,asset_param_2)=
    preconditions:
        quality:asset_param_1,quality_1=value_1;
        topology:asset_param_1->asset_param_2,topology_1;
    postconditions:
        delete topology:asset_param_1->asset_param_2,topology_1;
        insert quality:asset_param_1,quality_1=value_2;
.
\end{lstlisting}

Before continuing, a few words on the semantics of exploit processing may reduce
the chance of confusion. The \texttt{insert} operation inserts a new rule in the
fact base, which has the effect of replacing any previous rule with which it
conflicts. For example, the insertion of \texttt{value\_2} as the value of the quality
\texttt{quality\_1} in the example exploit \texttt{exploit\_1} would replace the
existing value of \texttt{value\_1} specified in the preconditions; no ambiguity
is introduced. The same is true of topologies.

Furthermore, the model has no innate distinction between one-way and two-way
topologies except the \texttt{<->} shorthand for specifying symmetric topologies.
That is to say, a bidirectional topology fact is actually implemented as
two unidirectional topology facts. Therefore, for example, if a bidirectional
topology exists in the fact base, one of its component directional topologies may
be removed by the realization of an exploit.
\TUsubsection{Generation Process}
This thesis is primarily concerned with modeling attack graphs, but it is worth
giving some consideration to the attack graph generation process used in its
reference implementation.

Currently attack graph generation proceeds according to a straightforward process
depending upon a monotonicity assumption: the attacker never moves ``backwards''.
That is, once an exploit is realized, even though the attacker may have the capability
of undoing it, he does not do so. The implementation used for this thesis is written
in Erlang and makes extensive use of Erlang's built-in \texttt{digraph} (its directed 
graph module),
used to represent the attack graph itself, and \texttt{ets} (Erlang term storage, which
provides constant-time table lookups) to associate graph vertices with network state data.

A maximum attack graph ``depth'' (really maximum permitted shortest path length from the
node representing the initial state)
is selected before generation begins. Then the program initializes a graph with a single
node (the starting state), and places the starting state and the node identifier in a
lookup table. Then it begins to recurse using the depth value, which is initialized to
the value selected by the user; and a list of available 
states (nodes), which is initialized to contain a single entry: the starting state.

The generation process is as follows. For each available state, a list of all valid 
attacks (or bound exploit patterns) is generated. 
Every permutation of every possible exploit realization from the state according to
the pattern preconditions is included in this attack list. The attacks' postcondition
operations are applied to the network state in order to generate (or load from memory, if
the new fact base has already been generated by another attack on this or another state)
successor states and add directed edges from the state in question to its successor state.
If the successor state is newly generated, it is added to a successor state list.
The generation function is called again, with the depth value decremented and the successor
state list passed as the new list of available states. Execution ceases when the depth
value reaches zero or the list of available states is empty.
% TODO: figure/flowchart?
\TUsubsection{Output}
In order to aid understanding the generation process, this section expands on the
illustrative example introduced in the preceding sections and demonstrates a
sample attack graph generation process using the example. The reader is warned to
avoid searching for meaning or design in the selection of these assets, qualities,
and topologies; they are intended to be illustrative only, and any relation to 
real world networks, problems, attacks, or situations is purely coincidental.
This section will use the following network model specification.
\begin{lstlisting}
network model = 
    assets :
        asset_1;
        asset_2;
        asset_3;

    facts :
        quality:asset_1,quality_1=value_1;
        quality:asset_2,quality_1=value_2;        
        quality:asset_3,quality_1=value_2;
        topology:asset_1->asset_2,topology_1;
        topology:asset_2<->asset_3,topology_2;
.
\end{lstlisting}

The following exploit pattern specifications are used in this section.
\begin{lstlisting}
exploit exploit_1(asset_param_1,asset_param_2)=
    preconditions:
        quality:asset_param_1,quality_1=value_1;
        topology:asset_param_1->asset_param_2,topology_1;
    postconditions:
        delete topology:asset_param_1->asset_param_2,topology_1;
        insert quality:asset_param_1,quality_1=value_2;
.

exploit exploit_2(asset_param_1,asset_param_2)=
    preconditions:
        quality:asset_param_1,quality_1=value_2;
        quality:asset_param_2,quality_1=value_2;
    postconditions:
        insert topology:asset_param_1<->asset_param_2,topology_2;
.
\end{lstlisting}

Fig.~\ref{fig:ill_topology_1} represents the initial network state 
(denoted State 1) specified
in this example. Note that Fig.~\ref{fig:ill_topology_1} is \emph{not} an attack
graph, merely a convenient graph based representation of the example network
in use here. Since there is only one quality in use in the entire example, its
name is not included in this graph representation; instead, each asset is labeled
with its identifier and the value of \texttt{quality\_1} with the prefixing 
\texttt{value\_} removed for brevity. Likewise, the edges that represent topologies
are labeled with the topology name they represent with the prefixing \texttt{topology\_}
removed.
\begin{figure}
\centering
\begin{dot2tex}[options=-t raw --autosize]
digraph G {
    rankdir=LR;
    asset_1 [shape=circle, texlbl="\begin{tabular}{c}\texttt{\bf asset\_1} \\ $1$ \end{tabular}"];
    asset_2 [shape=circle, texlbl="\begin{tabular}{c}\texttt{\bf asset\_2} \\ $2$ \end{tabular}"];
    asset_3 [shape=circle, texlbl="\begin{tabular}{c}\texttt{\bf asset\_3} \\ $2$ \end{tabular}"];
	asset_1 -> asset_2 [label=" ", texlbl="1"];
    asset_2 -> asset_3 [label=" ", texlbl="2"];
    asset_3 -> asset_2 [label=" ", texlbl="2"];
}
\end{dot2tex}
\caption{State 1 (initial network state) of the illustrative discrete example}
\label{fig:ill_topology_1}
\end{figure}

Execution of the generation process begins by specifying a maximum ``depth''
of generation, which will be 3 for the purposes of this exercise (although,
as this attack graph converges with maximum shortest path of 3 from the
initial state, this limitation is unnecessary), and by creating an initial list
of states for analysis, which contains only State 1.

Generation of the next set of states begins by creating a list of all valid
attacks on State 1; that is, selecting all valid bindings of assets given State 1's
fact base to exploit pattern parameters given their preconditions. Three such bindings
are possible from State 1: \texttt{exploit\_1(asset\_1, asset\_2)}; \texttt{exploit2(asset\_2, asset\_3)}; and
\texttt{exploit2(asset\_2, asset\_3)}. The latter two insert a topology into the fact base
that already exists, therefore generating State 1, so State 1 has itself as a successor state
twice. As State 1 already exists, these edges are added, and no further processing due to
those transitions is done. The first attack, \texttt{exploit\_1(asset\_1, asset\_2)}, results in
a new state, designated State 2. It is generated by performing the operations on State 1's
fact base and creating a new state based on the results. State 2 is the only state added to
the list of successor states.

\begin{figure}
\centering
\begin{dot2tex}[options=-t raw --autosize]
digraph G {
    rankdir=LR;
    asset_1 [shape=circle, texlbl="\begin{tabular}{c}\texttt{\bf asset\_1} \\ $2$ \end{tabular}"];
    asset_2 [shape=circle, texlbl="\begin{tabular}{c}\texttt{\bf asset\_2} \\ $2$ \end{tabular}"];
    asset_3 [shape=circle, texlbl="\begin{tabular}{c}\texttt{\bf asset\_3} \\ $2$ \end{tabular}"];
    asset_2 -> asset_3 [label=" ", texlbl="2"];
    asset_3 -> asset_2 [label=" ", texlbl="2"];
}
\end{dot2tex}
\caption{State 2 of the illustrative discrete example}
\label{fig:ill_topology_2}
\end{figure}

For the next execution of the generation function, the successor state list becomes
the new analysis state list, consisting now only of State 2, and the 
remaining allowed ``depth'' is decremented to 2. 6 attacks are possible from State 2:
\texttt{exploit\_2(asset\_1, asset\_2)}, \texttt{exploit\_2(asset\_2, asset\_1)}, 
\texttt{exploit\_2(asset\_1, asset\_3)}, \texttt{exploit\_2(asset\_3, asset\_1)},
\texttt{exploit\_2(asset\_2, asset\_3)}, and \texttt{exploit\_2(asset\_3, asset\_2)}. The process
iterates through these attacks, generating new states and, if they are new, adding them
to the list of successor states. \texttt{exploit\_2(asset\_1, asset\_2)} results in a new
state, State 3 (see Fig.~\ref{fig:ill_topology_3}), which is added to the list of
successor states. \texttt{exploit\_2(asset\_2, asset\_1)} also results in State 3, which
already exists and is therefore not added to the list of successor states. 
\texttt{exploit\_2(asset\_1, asset\_3)} generates a new state, State 4 
(see Fig.~\ref{fig:ill_topology_4}), which is added to the list of successor states.
\texttt{exploit\_2(asset\_3, asset\_1)} also generates State 4.
\texttt{exploit\_2(asset\_2, asset\_3)} and \texttt{exploit\_2(asset\_3, asset\_2)}
both result in State 2, which exists and is not added to the successor states.

\begin{figure}
\centering
\begin{dot2tex}[options=-t raw --autosize]
digraph G {
    rankdir=LR;
    asset_1 [shape=circle, texlbl="\begin{tabular}{c}\texttt{\bf asset\_1} \\ $2$ \end{tabular}"];
    asset_2 [shape=circle, texlbl="\begin{tabular}{c}\texttt{\bf asset\_2} \\ $2$ \end{tabular}"];
    asset_3 [shape=circle, texlbl="\begin{tabular}{c}\texttt{\bf asset\_3} \\ $2$ \end{tabular}"];
    asset_2 -> asset_3 [label=" ", texlbl="2"];
    asset_1 -> asset_2 [label=" ", texlbl="2"];
    
    asset_3 -> asset_2 [label=" ", texlbl="2"];
    asset_2 -> asset_1 [label=" ", texlbl="2"];
}
\end{dot2tex}
\caption{State 3 of the illustrative discrete example}
\label{fig:ill_topology_3}
\end{figure}

\begin{figure}
\centering
\begin{dot2tex}[options=-t raw --autosize]
digraph G {
    rankdir=LR;
    asset_1 [shape=circle, texlbl="\begin{tabular}{c}\texttt{\bf asset\_1} \\ $2$ \end{tabular}"];
    asset_2 [shape=circle, texlbl="\begin{tabular}{c}\texttt{\bf asset\_2} \\ $2$ \end{tabular}"];
    asset_3 [shape=circle, texlbl="\begin{tabular}{c}\texttt{\bf asset\_3} \\ $2$ \end{tabular}"];
    asset_2 -> asset_3 [label=" ", texlbl="2"];
    asset_1 -> asset_3 [label=" ", texlbl="2"];
    
    asset_3 -> asset_2 [label=" ", texlbl="2"];
    asset_3 -> asset_1 [label=" ", texlbl="2"];
}
\end{dot2tex}
\caption{State 4 of the illustrative discrete example}
\label{fig:ill_topology_4}
\end{figure}

The next iteration finds the remaining allowed depth at 1 and the list of
analysis states to contain State 3 and State 4. From State 3, the
possible attacks are the same as the possible attacks in State 2:
\texttt{exploit\_2(asset\_1, asset\_2)}, \texttt{exploit\_2(asset\_2, asset\_1)},
\texttt{exploit\_2(asset\_1, asset\_3)}, \texttt{exploit\_2(asset\_3, asset\_1)},
\texttt{exploit\_2(asset\_2, asset\_3)}, and \texttt{exploit\_2(asset\_3, asset\_2)}.
Only the middle two, \texttt{exploit\_2(asset\_1, asset\_3)}, \texttt{exploit\_2(asset\_3, asset\_1)},
generate a state that is not State 3 itself: State 5 (see Fig.~\ref{fig:ill_topology_5}),
which is added to the list of successor states. The rest simply create new reflexive
edges on State 3. The list of possible attacks on State 4 is the same, with only
\texttt{exploit\_2(asset\_1, asset\_2)} and \texttt{exploit\_2(asset\_2, asset\_1)} generating
a state other than State 4 itself. Both of those attacks generate State 5 again, which
already exists and therefore is not added to the list of successor states.

\begin{figure}
\centering
\begin{dot2tex}[options=-t raw --autosize]
digraph G {
    rankdir=LR;
    asset_1 [shape=circle, texlbl="\begin{tabular}{c}\texttt{\bf asset\_1} \\ $2$ \end{tabular}"];
    asset_2 [shape=circle, texlbl="\begin{tabular}{c}\texttt{\bf asset\_2} \\ $2$ \end{tabular}"];
    asset_3 [shape=circle, texlbl="\begin{tabular}{c}\texttt{\bf asset\_3} \\ $2$ \end{tabular}"];
    asset_1 -> asset_2 [label=" ", texlbl="2"];
    asset_2 -> asset_3 [label=" ", texlbl="2"];
    asset_3 -> asset_1 [label=" ", texlbl="2"];
    
    asset_2 -> asset_1 [label=" ", texlbl="2"];
    asset_3 -> asset_2 [label=" ", texlbl="2"];
    asset_1 -> asset_3 [label=" ", texlbl="2"];
}
\end{dot2tex}
\caption{State 5 of the illustrative discrete example}
\label{fig:ill_topology_5}
\end{figure}

The next iteration finds the list of analysis states to contain only State 5, and
the remaining allowed depth is 0. Because the depth limit is reached, generation
ceases. However, even if the depth limit had not been reached, execution would have
soon terminated. All 6 possible attacks from State 5 would have generated State 5 itself.
Therefore the list of successor states would be empty, and the next iteration's list of
analysis states would be empty, which would also cause generation to cease. A representation
of the resulting attack graph (with execution halting after 3 iterations) is provided
in Fig.~\ref{fig:ill_attack_graph}.

\begin{figure}
\centering
\begin{dot2tex}[options=-t raw --autosize]
digraph G {
    rankdir=LR;
    state_1 [shape=circle, texlbl="State 1"];
    state_2 [shape=circle, texlbl="State 2"];
    state_3 [shape=circle, texlbl="State 3"];
    state_4 [shape=circle, texlbl="State 4"];
    state_5 [shape=circle, texlbl="State 5"];
    
    state_1 -> state_1;
    state_1 -> state_1;
    state_1 -> state_2;
    
    state_2 -> state_2;
    state_2 -> state_2;
    state_2 -> state_3;
    state_2 -> state_3;
    state_2 -> state_4;
    state_2 -> state_4;
    
    state_3 -> state_3;
    state_3 -> state_3;
    state_3 -> state_3;
    state_3 -> state_3;
    state_3 -> state_5;
    state_3 -> state_5;
    
    state_4 -> state_4;
    state_4 -> state_4;
    state_4 -> state_4;
    state_4 -> state_4;
    state_4 -> state_5;
    state_4 -> state_5;
}
\end{dot2tex}
\caption{The illustrative example's attack graph to depth 3}
\label{fig:ill_attack_graph}
\end{figure}
\TUsection{Working Lexicon}
\TUsubsection{Introduction}
Because of the unrestricted nature of the terms available to a modeler
in this version of the attack graph framework, in order to proceed systematically
some conventions must be established in the use of terms. This thesis recommends
a set of conventions designed to ease the goal of automated exploit pattern extraction
from the National Vulnerability Database (NVD) maintained by the 
National Institute of Standards and Technology (NIST), which among other roles
indexes MITRE's Common Vulnerability Enumeration (CVE). ``NVD includes 
databases of security checklists, security related software flaws, 
misconfigurations, product names, and impact metrics''~\cite{nvdhome}.
\TUsubsection{National Vulnerability Database}
Several concepts are used across the NVD's index of vulnerabilities; they
strongly inform the working lexicon employed in this thesis and include
a vulnerability's access vector and impact type.
\TUsubsubsection{Access vector}
The access vector of a vulnerability on NVD refers to the logical location from
which the attacker may launch the attack. The possible attack vectors according
to the National Vulnerability Database are as follows.
\begin{description}
\item[Local] The vulnerability is exploitable through physical or local account access
    to the device on which the vulnerability resides.
\item[Adjacent Network] The vulnerability is exploitable through access to a network
    that is adjacent to the vulnerable host; that is, the attacker must be in the same
    broadcast domain or collision domain (e.g. the same network segment or VLAN).
\item[Network] No local or adjacent access is required to exploit the vulnerability;
    in other words, it is exploitable over the Internet.
\end{description}
\TUsubsubsection{Impact Type}
The vulnerability's impact type on NVD places the effects of the vulnerability's
exploitation on the target into one of the following categories.
, and another for some other type of privileged access.
\begin{description}
\item[Confidentiality] Confidentiality impacts allow the unauthorized disclosure 
    of information (corresponding to STRIDE's ``information disclosure'')
\item[Integrity] Allows modification of data (corresponding to STRIDE's ``tampering'')
\item[Availability] Availability impacts correspond to disruptions of service (STRIDE's
    ``denial of service'')
\item[Security Protection] The security protection category refers to the effects
    of exploits that provide unauthorized access to the target. This may be either
    general system access or application access. The security protection category
    corresponds roughly to STRIDE's ``elevation of privileges'' and also partly encompasses
    ``spoofing identity''. It has three subcategories:
    \begin{description}
    \item[User access] This subcategory refers to the attacker's gaining user level access
        to the operating system.
    \item[Administrative access] This subcategory refers to the attacker's gaining root level
        access to the operating system.
    \item[Other access] This subcategory refers to any other type of privileged access on
        the target.
    \end{description}
\end{description}
\TUsubsection{Topologies}
Introduction of actual terms begins with topologies, but first a note about the modeling
of adversaries is needed. 

Two approaches are possible for modeling the attacker. The first
is to design the network model so that it is, in a way, from the attacker's perspective.
In this scheme, henceforth the \emph{first person} strategy, the attacker is treated as
implicit, and its access and connections are considered properties of the system itself.
The adversary's access level to a server, for instance, would be modeled as a quality of
that server asset.

In the second approach, henceforth the \emph{third person} strategy, the attacker is
modeled as a first class part of the system: as an asset. Its properties, connections,
and access levels are modeled as qualities and topologies of the adversary asset.
This strategy, which is the one employed primarily in this thesis, can model multiple
adversaries. Aside from that difference, they are roughly equivalent and a matter
of taste on behalf of the modeler and analyst. This section's terminology, however,
assumes the third person model.

Two types of topology terms are introduced in this section: connection and access. These
correspond directly to the two NVD concepts introduced in the previous section.
\TUsubsubsection{Connection Topologies}
Connection topologies refer to how two assets are connected, over the network or otherwise.
They may be local, adjacent, or network connected. These connection topologies begin with
the word \texttt{connected}, followed by an underscore, then one of the three connection
types: \texttt{local}, \texttt{adjacent}, or \texttt{network}. For local and adjacent
connections, this is all that is necessary. 

For network connections it is
necessary to specify the available protocols individually. These are done by appending another
underscore, then the lower case version of the standard abbreviation for the protocol
(e.g. \texttt{connected\_network\_http} for web or \texttt{connected\_network\_ssh} for
secure shell).

Connections are one-way: the source of the connection topology is considered to be the
``client'' in the relationship, and the destination of the connection is considered
the ``server'', where such distinctions are meaningful.
\TUsubsubsection{Access Topologies}
Access topologies refer to trust relationships and distinguish what kind of access one
asset (possibly a program, individual, attacker, user, or other security principal) has
to another. Much like the subcategories of the Security Protection impact type,
there are three basic types of access topology (plus the lack of a topology, which
signifies a lack of any noteworthy trust or access relationship): user access,
root access, and other access.

They are named using the word \texttt{access}, followed by an underscore, followed
by the lowercase name of the access type, then, if the access type is \texttt{other},
an optional underscore delimited description of the access type (perhaps the name
of the application whose access level is in question). Examples include
\texttt{access\_user}, \texttt{access\_root}, \texttt{access\_other\_apache}, or
\texttt{access\_other\_vsftpd}.
\TUsubsection{Qualities}
Although qualities are used in a more \emph{ad hoc} fashion to describe entity-specific
asset properties, there is still a role for a 
\TUsection{Examples}
\TUsubsection{Blunderdome}
\TUsubsection{Denial of Sleep}
\TUsection{State Predicates}
\TUsection{State Aggregation}